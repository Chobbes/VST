% A few hacks for showing index entries in draft mode
%\setmarginnotes{20pt}{50pt}{10pt}
%\let\OldIndex\index
%\renewcommand{\index}[1]{{\textcolor{darkred}#1}\OldIndex{xx#1}}
% end of ``A few hacks''

\newcommand*{\xchapter}[1]{\refstepcounter{chapter}
  \ \newline
    {\rmfamily\itshape\Large\thechapter\hspace{1em} #1 }
  \newline
  \addcontentsline{toc}{chapter}{\protect\numberline{\thechapter}#1}}

\newcommand*{\xpart}[1]{\refstepcounter{part}
  \par\vspace*{20pt}
  \noindent
    {\rmfamily\itshape\huge PART~\thepart\hspace{1em} #1 }
  \newline
  \addcontentsline{toc}{part}{\protect\numberline{\thepart}#1}}

\newsavebox{\mybox}

\newcommand\thefile{}
\newcommand\includex[1]{\renewcommand\thefile{\small\textsf{#1}}\include{#1}}

\renewcommand{\_}{\texttt{\textunderscore}}

\newcommand{\measure}[3]{#1/#2$\times$\unit[#3]{pc}}

\newcommand{\hairsp}{\hspace{1pt}}% hair space
\newcommand{\ie}{\mbox{\textit{i.\hairsp{}e.}}\xspace}
\newcommand{\eg}{\mbox{\textit{e.\hairsp{}g.}}\xspace}
\newcommand{\opcit}{\emph{op. cit.}}

\newcommand{\emp}{\mathsf{emp}}
\newcommand{\subtype}{\hspace{1pt}\includegraphics[width=1em]{Tailrightarrow.pdf}\,}
\newcommand\rtarrow{\hspace{-.4ex}\rightarrow\hspace{-.5ex}}

\DeclareRobustCommand\longrightarrow
     {\relbar\mathrel{\mkern-4mu}\rightarrow}

\DeclareRobustCommand
  \longmapsto{\mapstochar\longrightarrow}


\renewcommand{\mapsto}{\,\includegraphics[width=.9em, trim=0 0 0 3]{mapsto.pdf}\,}
\newcommand{\lbul}{\includegraphics[width=.44em, trim=0 0 0 3]{lbul.pdf}\,}
\newcommand{\rbul}{\includegraphics[width=.44em, trim=0 0 0 3]{rbul.pdf}\,}

\newcommand{\triple}[3]{\{#1\}\,#2\,\{#3\}}
\mathlig{|->}{\mapsto}
\newcommand{\later}{\triangleright}
%\newcommand{\wand}{\mathrel{\relbar\joinrel\!\!\relbar\!\!\!\ast\,}}
\newcommand{\wand}{\mathrel{-\hspace{-.7ex}*}}
\newcommand{\ewand}{\mathrel{-\hspace{-.42ex}\circ}}
%\newcommand{\ewand}{\multimap}
\newcommand{\TT}{\ensuremath{\mathbf{\top}}}
\newcommand{\FF}{\ensuremath{\mathbf{\bot}}}
\renewcommand{\models}{\mathrel{|}\hspace{-.1ex}\joinrel\Relbar}
%\newcommand\guards{\raisebox{-.3ex}{\makebox[0pt][l]{$\sqcup$}}\raisebox{.2ex}{$\sqcap$}}
\newcommand\guards[2]{\{#1\}\hairsp #2}
\newcommand\semaxfunc[3]{#1\vdash_\mathrm{func} #2:#3}
\newcommand\pair[2]{\left<#1,#2\right>}
\newcommand{\boxdotright}{\!\mathrel\boxdot\joinrel\rightarrow\!}
\newcommand{\islock}{\boxdotright}
\newcommand{\islocksh}[1]{\stackrel{#1}\boxdotright}
\newcommand{\nxt}[2]{\mathsf{next}\,#1\,#2}
\newcommand{\lseg}[2]{#1\!\! \leadsto \!\!#2}
\newcommand\EVAL[2]{[\hspace{-.25em}[ #1 ]\hspace{-.25em}]_#2}
\newcommand{\listrep}[3]{#2 \stackrel{#1}{\leadsto} #3}
\newcommand{\listrepsh}[4]{#3 \stackrel{#2}{\leadsto}_{#1} #4}
\newcommand{\capar}[1]{~{\scriptsize #1:}}
\newcommand{\typecheck}[2]{#1 \vdash_\mathrm{type} #2}
\newcommand{\expcheck}[2]{#1 \vdash_\mathrm{exp} #2}

\newcommand{\andp}{\ensuremath{\,\mathsf{\&\hspace{-.2em}\&}\,}}
\newcommand{\orp}{\ensuremath{\,\|\,}}
\newcommand{\imp}{\ensuremath{\longrightarrow}}
\newcommand{\prop}{\ensuremath{!!}}
\newcommand\fash{\ensuremath{\#}}
\newcommand{\unfash}{!}

\newcommand{\PROP}{\mbox{\small PROP}}
\newcommand{\LOCAL}{\mbox{\small LOCAL}}
\newcommand{\SEP}{\mbox{\small SEP}}


\newcommand\oo{\circ}
\newcommand\Prop{\mathsf{Prop}}

\newcommand{\file}[1]{\mbox{\textsf{#1}}\index{#1@\textsf{#1}}}
\newcommand{\idef}[1]{\mbox{\textsf{#1}}\index{#1@\textsf{#1}}}
\newcommand{\iref}[1]{\mbox{\textsf{#1}}\index{#1@\textsf{#1}}}
\newcommand{\indexx}[1]{#1\index{#1}}
\newcommand{\indexy}[1]{\index{#1}}

\IfFileExists{redindex.tex}{\input{redindex.tex}}{}

\newcommand{\defeq}{~\mathsf{:=}~}
\renewenvironment{prooftree}%
{\par \vskip2ex plus.8ex minus.4ex}%
{\DisplayProof \par  \vskip2ex plus.8ex minus.4ex }

%% Set up the Qtree stuff
\renewcommand{\qtreepadding}{2pt}
\renewcommand{\qtreeunaryht}{0.5em}
\newcommand{\qleafhook}{}

\newcommand{\inl}{\mathrm{inl}}
\newcommand{\inr}{\mathrm{inr}}
\newcommand{\inj}{\mathrm{inj}}
\newcommand{\sa}[2]{\langle #1, #2 \rangle}
\newcommand{\powerset}{\mathcal{P}}
\newcommand{\Lbrack}{\llbracket}
\newcommand{\Rbrack}{\rrbracket}

\newcommand{\lf}[1]{\raisebox{.5ex}[.5ex]{#1}}
\newcommand{\lfx}[1]{\raisebox{.8ex}[.2ex]{#1}}
\newcommand{\TinyTree}[1]
  {\renewcommand{\qleafhook}{\small}
   \raisebox{-.5ex}[0em][0em]{#1}
   \renewcommand{\qleafhook}{}}
\newcommand{\T}{\bullet}
\newcommand{\F}{\circ}
\newcommand{\Troot}
  {\TinyTree{\Tree [ $\T$ ] }}
\newcommand{\Froot}
  {\TinyTree{\Tree [ $\F$ ] }}
\newcommand{\splitLeft}
  {\TinyTree{\Tree [ \lfx{$\T$} \lfx{$\F$} ] }}
\newcommand{\splitRight}
  {\TinyTree{\Tree [ \lfx{$\F$} \lfx{$\T$} ] }}

\newcounter{save_eqn}

%%
% Prints argument within hanging parentheses (i.e., parentheses that take
% up no horizontal space).  Useful in tabular environments.
\newcommand{\hangp}[1]{\makebox[0pt][r]{(}#1\makebox[0pt][l]{)}}

%%
% Prints an asterisk that takes up no horizontal space.
% Useful in tabular environments.
\newcommand{\hangstar}{\makebox[0pt][l]{*}}

\definecolor{shadecolor}{cmyk}{0.10,0.03,0,0}

\hyphenation{Comp-Cert}
\renewcommand{\partpageend}{\vspace{2\baselineskip}}

\newcommand{\estimatedpages}[1]{
\par
 \emph{An estimated #1 pages of material goes here.}
%\addtocounter{page}{#1}\addtocounter{page}{-1}
}


%%% Syntax
\newcommand{\keyw}[1]{\mathsf{#1}}
\newcommand{\option}[1]{\keyw{option}\;#1}
\newcommand{\List}[1]{\keyw{list}\;#1}
%\newcommand{\Type}{\keyw{Type}}
%\newcommand{\Prop}{\mathbb{T}}
\newcommand{\Some}[1]{\keyw{Some}\ #1}

\newcommand{\valtype}{\mathcal{V}}
\newcommand{\eftype}{\mathcal{F}}


\newcommand{\Src}{\ensuremath{\mathrm{Src}}}
\newcommand{\Mid}{\ensuremath{\mathrm{Mid}}}
\newcommand{\Tgt}{\ensuremath{\mathrm{Tgt}}}
\newcommand{\fwdsimEq}{\ensuremath{\cong}}
\newcommand{\fwdsimExt}{\ensuremath{\simeq}}
\newcommand{\fwdsimInj}[1]{\ensuremath{\approx}_{#1}}
%\newcommand{\inject}[3]{\ensuremath{{#2} \xhookrightarrow{\ {#1}\ } {#3}}}
\newcommand{\extend}[2]{\ensuremath{{#1} \rightarrowtail {#2}}}

\newcommand\Prefix[3]{\vphantom{#3}#1#2#3}
\newcommand{\injectseparated}[4]{\ensuremath{{#1}\ \Prefix_{{#3}}\!{\bowtie}_{{#4}} {{#2}}}}
\newcommand{\dom}[1]{\ensuremath{\mathrm{dom}({#1})}}
